\section{Mathematical Model}
\label{sec:mathmod}

To model these systems \textit{in silico},

\subsubsection{Governing Equations}
\label{subsub:goveq}

The frequency domain Maxwell's Equations in the absence of electric or fictitious magnetic currents are,
\begin{align}
	\nabla \times \textbf{E} = -j\omega\textbf{B},
	\label{eq:faraday}
\end{align}
and
\begin{align}
	\nabla \times \textbf{H} = -j\omega\textbf{D}
	\label{eq:ampere}
\end{align}
where \textbf{E} is the electric field intensity, \textbf{B} is the magnetic flux density, \textbf{H} is the magnetic field intensity, and \textbf{D} is the electric flux density.

For a homogenous, infinite waveguide filled with a non-dispersive dielectric, \textbf{B} and \textbf{D} can be rewritten as
\begin{align}
	\textbf{B} = \mu \textbf{H},
	\label{eq:corH}
\end{align}
and
\begin{align}
	\textbf{D} = \epsilon \textbf{E}.
	\label{eq:corE}
\end{align}

These constitutive relations can now be used to simplify (\ref{eq:faraday}-\ref{eq:ampere}) as in 
\begin{align}
	\nabla \times \textbf{E} = -j\omega\mu\textbf{H},
	\label{eq:faraday_reduced}
\end{align}
and
\begin{align}
	\nabla \times \textbf{H} = -j\omega\epsilon\textbf{E}.
	\label{eq:ampere_reduced}
\end{align}

In the case of the infinite waveguide, the TM and TE modes can be fully solving for $E_z$ and $H_z$ respectively as all other field components can be derived from these two transverse fields \cite{rothlecnotes}-\cite{jin2011theory}. With this, (\ref{eq:faraday_reduced}-\ref{eq:ampere_reduced}) can be manipulated to solve for two independent 2-dimensional Helmholtz equations as
\begin{align}
	\nabla_t^2 E_z+k_c^2E_z=0 \quad \mathrm{on} \ \Omega,
	\label{eq:tm_helmholtz}
\end{align}
and
\begin{align}
	\nabla_t^2 H_z+k_c^2H_z=0 \quad \mathrm{on} \ \Omega
	\label{eq:te_helmholtz}
\end{align}
where $\nabla_t^2=\partial^2_x+\partial^2_y$ the the trasverse Laplacian operator in cartesian coordinates, $k_c^2=\omega^2\mu\epsilon+k_z^2$ is the cutoff wave number, $k_z$ is the wavenumber in the direction of propagation, and $\Omega$ denotes all non-boundary locations within the simulation domain.

These relations hold for all locations excluding those on the PEC walls of the waveguide. This PEC wall condition manifests in the form of a Dirichlet boundary condition
\begin{align}
	E_z=0 \quad \mathrm{on} \ \partial\Omega
	\label{eq:diriclet_tm}
\end{align}
for the TM mode and Neumann boundary conditions
\begin{align}
	\partial_x H_z = 0,\ \partial_y H_z = 0 \quad \mathrm{on} \ \partial\Omega
\end{align}
for the TE mode where $\partial\Omega$ denotes the PEC surface surrounding the waveguide.

\subsubsection{Galerkin Weak Formulation}
\label{subsub:galerkin_weak}


\subsubsection{Finite Element Matrix Assembly}
\label{subsub:mat_assembly}