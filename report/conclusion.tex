\section{Conclusion}
\label{sec:conclusion}
A 3-dimensional finite difference time domain was developed from Maxwell's Equations for a rectangular waveguide and cavity resonator. The model was validated against analytic results for narrow and wide band signals thereby verifying the model's calculated fields. From this, several dielectric materials were compared for use in X-Band cavity resonators at $10$GHz. These compared results were then explained using theoretical unloaded quality factors vurther verifying the accuracy of the model.

While relatively performant, there are many optimizations that could be made to the underlying implementation. Most notably tiled approaches could be taken to improve program cache locality to alleviate the memory bound nature of the loops in this implementation. Tiled approaches would also aid in exploiting the embarrassingly parallel structure Yee's FDTD algorithm gives rise to. Further improvements could also be made to the implementation to allowing for more complex geometries to be represented which may be useful for placing devices inside waveguides or using the waveguide as a source for another device. Finally, the user experience of this implementation should be improved as it is remarkably easy to save in tens to hundreds of gigabytes of data inadvertently shifting the bottleneck away from memory to disk performance.