\section{Conclusion}
\label{sec:conclusion}
A 2-dimensional finite element method program was developed from Maxwell's Equations allowing for field distribution and dispersion analysis of arbitrarily shaped homogenous waveguides. The model was first validated against the analytic field profiles and dispersion curves for a rectangular waveguide. From here, additional verification was performed on the field profiles of circular waveguides. Next, the model was used to predict the dispersion characteristics of said circular guide which can be difficult to do analytically without tabulated Bessel function data. Finally, the model was used to compare ridged waveguides to their non-ridged counterparts. It was shown that the inclusion of ridges significantly expands the bandwidth available to the dominant $\mathrm{TE}_{10}$ mode in alignment with theory. This phenomenon was explained using the change in field buckling caused by complex modes requiring higher frequencies to achieve the same field profiles where as the buckling working in the favor of simpler field profiles resulting in a lowering of the cutoff wave number.

While relatively general in the sense that the model works for any \verb|.inp| input waveguide mesh, the model would benefit from increased work regarding the identification of modes which currently is done manually by the user in order to produce dispersion curves and field profile plots. The model would also benefit from using sparse matrix types as defined in SciPy as opposed to the dense NumPy arrays used in this program. In addition to this, it would be interesting to combine this model with the FDTD model developed in the last project in order to generate the field profiles of more complex waveguides for use in the TF/SF source condition. This would allow for temporal analysis of more sophisticated waveguides than the simple rectangular model used in the former project. 