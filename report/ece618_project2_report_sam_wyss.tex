% This is a comment
\documentclass[journal]{IEEEtran}
% IEEEtran.cls must be in the same folder.
% If not, manually specify the path to it like:
% \documentclass[conference]{../sty/IEEEtran}

% This is the somewhat "standard" set of packages that I typically include in all my documents. Some of them probably aren't actually used, so it isn't "pretty" but it gets the job done.
\usepackage{array}
\usepackage{url}
\usepackage{epsf}
\usepackage{verbatim}
\usepackage{morefloats}
\usepackage{flushend}
\usepackage[cmex10]{amsmath}
\usepackage{esint}
\usepackage{amssymb}
\usepackage{graphicx}
\usepackage{hhline}
\usepackage{longtable}
\usepackage{rotating}
\usepackage{lscape}
\usepackage{psfrag}
\usepackage[labelformat=simple]{subcaption}
\renewcommand\thesubfigure{(\alph{subfigure})} % makes subfigure labels appear as (a) and (b) and referenced like this in the text
\usepackage{epstopdf}
\usepackage{cite}
\usepackage{booktabs} %professional-looking tables
\usepackage{multicol} %used for getting multicolumn without page-break
\usepackage{multirow} %used for multi row in tables
\usepackage{setspace}
\usepackage[dvipsnames]{xcolor}
\usepackage{mathtools}
\usepackage{textcomp}
\newcommand{\textapprox}{\raisebox{0.5ex}{\texttildelow}} % defines a new command to make the ~ in text to denote "approximately"


% I like to keep my images that I will include as figures organized in a subfolder names "figures". 
% Adjust this appropriately for your use by either changing the path or by commenting it out if you just want to leave all the figures in the same folder as your main file.
\graphicspath{{./figures/}} 


\begin{document}
% paper title
% can use linebreaks \\ within to get better formatting as desired
% Generally good to not put math or special symbols in the title.
\title{Finite Element Comparison of Homogenous Ridged and Non-Ridged X-Band Rectangular Waveguide Dispersion Characteristics}

% I'm mashing some templates together here. If you write an actual IEEE transactions paper in LaTeX start from their template and then modify things like packages and files you want to include from there as opposed to starting from this document provided in this course.
\author{
\IEEEauthorblockN{Samuel J. Wyss\IEEEauthorrefmark{2} }
\IEEEauthorblockA{\\ \IEEEauthorrefmark{2}School of Nuclear Engineering\\
Purdue University\\
West Lafayette, Indiana 47907 \\ E-mail: wysss@purdue.edu}\\
}

% make the title area
\maketitle

% As a general rule, do not put math, special symbols or citations
% in the abstract
\begin{abstract}
Two dimensional Finite Element Analysis (FEA) is applied to assess dispersion characteristics of homogenous rectangular, circular and ridged rectangular X-Band waveguides. To model these systems \textit{in silico}, the weak form of the wave equation is derived from Maxwell's Equations for both TE and TM modes. Perfect electrical conductors (PECs) are used as waveguide walls as to neglect the effect of wave leakage into the environment. The model is validated against the analytical dispersion curves for homogenous rectangular waveguides. Dispersion characteristics of circular waveguides are assessed. A comparison of dispersion characteristics for ridged and non-ridged rectangular waveguides is provided which is then used to assess real world applications of ridged waveguides. 
\end{abstract}

% For peerreview papers, this IEEEtran command inserts a page break and
% creates the second title. It will be ignored for other modes.
\IEEEpeerreviewmaketitle

% the \input command allows you to tell the compiler to go through the .tex file placed in the command. Breaking out your main sections of the paper in this way helps keep the code/tex document more organized typically.
%\section{Getting Started}
\label{sec:getting-started}
Before you can start using LaTeX you need to get your computer set up for it. If you don't want to go through the process of installing a few different programs on your computer, you can sign up to use Overleaf. This is an online LaTeX editor that Purdue students and faculty have access to for free. It is fairly easy to use and has nice features to enable collaborating on documents.

If you want to use LaTeX on your computer, you first need to install MiKTeX. If you google it you should be able to quickly find the download instructions. Once you have that installed, I recommend using an additional ``editor'' on top of that which has additional features to improve the usability. I personally use TeXstudio because that is what was recommended to me initially and I haven't spent any time shopping around for a different tool because it seems straightforward and capable enough.

Note that I am using the IEEE template for formatting in this document. You can easily swap to a different journal's template by substituting in their class file (replacing the IEEEtran file with something else) and making a few other edits to the main TeX document. This kind of swap can be a little tedious, but it is typically substantially easier than trying to swap between two Word document templates from different journals. I will say that the IEEE template can be a little annoying with formatting when the document you are writing only has a small amount of content in it. It usually starts to do a better job once the document gets filled in more, but if you have lots of long equations it can still struggle. These issues are why there is occasionally some weird spacing between certain sections in this document. LaTeX is doing its best to try and move things around on the pages to make it look good and match the IEEE format, but sometimes more control of these issues can be needed. I usually leave this to the editorial staff at a journal to fix, since they are going to mess with whatever you submit most of the time anyway.
\section{Introduction}
\label{sec:intro}

Waveguides are used in a plethora of applications ranging from transmitting microwave fields to acting as passive, low-pass filters \cite{pozar2011microwave}. While any cross section of a single conductor waveguide can support TE and TM modes, rectangular and circular cross sections are commonly chosen due to their ease of construction and analytic propagation characteristics. However, a limitation of rectangular waveguides is the constrained bandwidth of their dominant mode, which is less than an octave \cite{pozar2011microwave}. By adding a single or double-ridge to the mouth of a waveguide, the cutoff frequency of the dominant mode can be reduced thus allowing for increased signal bandwidth \cite{pozar2011microwave}. This increased bandwidth comes at the cost of reduced power capacity due to the reduction in breakdown potential between the ridges \cite{pozar2011microwave}, making ridged waveguides less ideal for High Power Microwave (HPM) devices.  

All wave phenomena in an arbitrarily shaped, infinitely long waveguide at a given frequency are governed by the frequency domain Maxwell's Equations. Of these equations, Faraday's and Amp\`{e}re's laws can be manipulated to create Helmholtz wave equations which capture nearly all electromagnetic wave phenomenon to a high degree of accuracy \cite{rothlecnotes}. The Finite Element Method (FEM) converges on the analytic solution of the wave equation by approximating a weak form of the Helmholtz equations over a finite set of elements within the simulation domain using weighted residuals. To convert the full wave equation to its weak form, the Galerkin method is employed for which the weighting functions are identical to continuous basis functions as is common in Computational Electromagnetics (CEM) Finite Element Analyis (FEA) \cite{rothlecnotes}. In the case of an arbitrarily shaped, infinitely long waveguide, the full 3D, frequency-domain solutions can be obtained by solving for the fields in a cross sectional slice of the waveguide, which is quite efficient. FEM operates on non-uniform, conformal meshes, allowing for arbitrary waveguide cross sections to be modeled without stair-stepping error like that of the structured meshes of Finite-Difference Time-Domain (FDTD).

The development and results of this work are laid out as follows. Section \ref{sec:mathmod} contains derivations of the Galerkin weak forms of the Helmholtz wave equations for both the TE and TM modes as well as the formation of the FEM matrices via the assembly process. Section \ref{sec:numres} contains a verification of the model with analytic data for a square waveguide, an analysis and discussion of propagation in circular waveguides, and a comparison of rectangular waveguides to their ridged counterparts. Finally, Section \ref{sec:conclusion} contains closing remarks regarding the analysis and potential future work.
\section{Mathematical Model}
\label{sec:mathmod}

To model these systems \textit{in silico},

\subsubsection{Governing Equations}
\label{subsub:goveq}

The frequency domain Maxwell's Equations in the absence of electric or fictitious magnetic currents are,
\begin{align}
	\nabla \times \textbf{E} = -j\omega\textbf{B},
	\label{eq:faraday}
\end{align}
and
\begin{align}
	\nabla \times \textbf{H} = -j\omega\textbf{D}
	\label{eq:ampere}
\end{align}
where \textbf{E} is the electric field intensity, \textbf{B} is the magnetic flux density, \textbf{H} is the magnetic field intensity, and \textbf{D} is the electric flux density.

For a homogenous, infinite waveguide filled with a non-dispersive dielectric, \textbf{B} and \textbf{D} can be rewritten as
\begin{align}
	\textbf{B} = \mu \textbf{H},
	\label{eq:corH}
\end{align}
and
\begin{align}
	\textbf{D} = \epsilon \textbf{E}.
	\label{eq:corE}
\end{align}

These constitutive relations can now be used to simplify (\ref{eq:faraday}-\ref{eq:ampere}) as in 
\begin{align}
	\nabla \times \textbf{E} = -j\omega\mu\textbf{H},
	\label{eq:faraday_reduced}
\end{align}
and
\begin{align}
	\nabla \times \textbf{H} = -j\omega\epsilon\textbf{E}.
	\label{eq:ampere_reduced}
\end{align}

In the case of the infinite waveguide, the TM and TE modes can be fully solving for $E_z$ and $H_z$ respectively as all other field components can be derived from these two transverse fields \cite{rothlecnotes}-\cite{jin2011theory}. With this, (\ref{eq:faraday_reduced}-\ref{eq:ampere_reduced}) can be manipulated to solve for two independent 2-dimensional Helmholtz equations as
\begin{align}
	\nabla_t^2 E_z+k_c^2E_z=0 \quad \mathrm{on} \ \Omega,
	\label{eq:tm_helmholtz}
\end{align}
and
\begin{align}
	\nabla_t^2 H_z+k_c^2H_z=0 \quad \mathrm{on} \ \Omega
	\label{eq:te_helmholtz}
\end{align}
where $\nabla_t^2=\partial^2_x+\partial^2_y$ the the trasverse Laplacian operator in cartesian coordinates, $k_c^2=\omega^2\mu\epsilon+k_z^2$ is the cutoff wave number, $k_z$ is the wavenumber in the direction of propagation, and $\Omega$ denotes all non-boundary locations within the simulation domain.

These relations hold for all locations excluding those on the PEC walls of the waveguide. This PEC wall condition manifests in the form of a Dirichlet boundary condition
\begin{align}
	E_z=0 \quad \mathrm{on} \ \partial\Omega
	\label{eq:diriclet_tm}
\end{align}
for the TM mode and Neumann boundary conditions
\begin{align}
	\partial_x H_z = 0,\ \partial_y H_z = 0 \quad \mathrm{on} \ \partial\Omega
\end{align}
for the TE mode where $\partial\Omega$ denotes the PEC surface surrounding the waveguide.

\subsubsection{Galerkin Weak Formulation}
\label{subsub:galerkin_weak}
With the governing equations established, we can now proceed with the discretization of an arbitrarily shaped waveguide to solve for both the transverse electric and magnetic fields. Using FEM, we break these 2D waveguide slices into a finite set of finitely sized elements and approximate the solution of (\ref{eq:tm_helmholtz}-\ref{eq:te_helmholtz}) over each element. Triangular elements are chosen as they can be meshed together to form the boundaries of arbitrarily curved shapes well making them ideal for modeling geometries with no analytic solutions \cite{jin2011theory}. Linearly interpolating functions are used to approximate the solution of (\ref{eq:tm_helmholtz}-\ref{eq:te_helmholtz}) at the nodes of each element. Linear interpolating functions are chosen for their overall simplicity and adequate accuracy for the determination of waveguide parameters \cite{rothlecnotes}\cite{jin2011theory}. 

For an arbitrary triangular element, an generic scalar field $\phi$ can be Linearly interpolated over using
\begin{align}
	\phi^{(e)}(x,y)=a+bx+cy
	\label{eq:basic-el-interp}
\end{align} 
where $(e)$ refers to a specific element, $a$, $b$, $c$ are scaling constants and $x$, $y$ are the coordinates of the location within the node \cite{jin2011theory}. This interpolation scheme can now be applied to find the field value at an arbitrary vertex on the element as
\begin{align}
	\phi^{(e)}_l=a+bx_l+cy_l
	\label{eq:node-interp}
\end{align}
where $x$, $y$ are the coordinates of the node \cite{jin2011theory}. These nodal field expressions can now be combined to rewrite (\ref{eq:basic-el-interp}) in terms of the field value calculated at each node as
\begin{multline}
	\phi^{(e)}(x,y)=N_1^{(e)}(x,y)\phi^{(e)}_1\\+N_2^{(e)}(x,y)\phi^{(e)}_2+N_3^{(e)}(x,y)\phi^{(e)}_3
	\label{eq:elementinterp}
\end{multline} with an arbitrary interpolating function $N_l^{(e)}$ given by 
\begin{align}
	N_l^{(e)}(x,y)=\frac{1}{2\Delta^{(e)}}\left(a_l^{(e)}+b_l^{(e)}x+c_l^{(e)}y\right)
\end{align}
where $\Delta^{(e)}$ is the area of element $e$ and $a_l^{(e)}$, $b_l^{(e)}$, $c_l^{(e)}$ are given by the following as in \cite{jin2011theory}
\begin{multline}
	\quad \ \ \ a_1^{e} = x^{e}_2y^{e}_3-x^{e}_3y^{e}_2,b_1^{e}=y^{e}_2-y^{e}_3, c_1^{e}=x^{e}_3-y^{e}_2
	\\
	a_2^{e} = x^{e}_3y^{e}_1-x^{e}_1y^{e}_3,b_2^{e}=y^{e}_3-y^{e}_1, c_2^{e}=x^{e}_1-y^{e}_3
	\\
	a_3^{e} = x^{e}_1y^{e}_2-x^{e}_2y^{e}_1,b_3^{e}=y^{e}_1-y^{e}_2, c_3^{e}=x^{e}_2-y^{e}_1.
	\label{eq:coefficients}
\end{multline}

With definitions (\ref{eq:elementinterp}-\ref{eq:coefficients}) an arbitrary field with potential Dirichlet boundary conditions (noted by $D$) can be expressed as the superposition of all fields at each node as 
\begin{align}
	\phi=\sum_{j=1}^{N}N_j\phi_j\sum_{j=1}^{N}N_j^D\phi_j^D
\end{align}

With the discretization of a generic field outlined, The weak forms of (\ref{eq:tm_helmholtz}-\ref{eq:te_helmholtz}) will now be derived in parallel.

\subsubsection{Finite Element Matrix Assembly}
\label{subsub:mat_assembly}
\section{Numerical Results}
\label{sec:numres} 
All update equations as defined in Section \ref{subsec:timestepeqs} were implemented in Rust. This language was chosen for its C++ like performance while enforcing compile-time memory safety which makes writing fast and safe CEM codes relatively easy. An overview of this implementation can be found in \ref{subsec:code}.

\subsection{Verification and Validation}
\label{subsec:vv}


\subsection{Circular Waveguides}
\label{subsec:circ_guides}

\subsection{Comparison of Ridged and Non-Ridged Waveguides}
\label{subsec:rid_guides}

\section{Conclusion}
\label{sec:conclusion}
A 3-dimensional finite difference time domain was developed from Maxwell's Equations for a rectangular waveguide and cavity resonator. The model was validated against analytic results for narrow and wide band signals thereby verifying the model's calculated fields. From this, several dielectric materials were compared for use in X-Band cavity resonators at $10$GHz. These compared results were then explained using theoretical unloaded quality factors vurther verifying the accuracy of the model.

While relatively performant, there are many optimizations that could be made to the underlying implementation. Most notably tiled approaches could be taken to improve program cache locality to alleviate the memory bound nature of the loops in this implementation. Tiled approaches would also aid in exploiting the embarrassingly parallel structure Yee's FDTD algorithm gives rise to. Further improvements could also be made to the implementation to allowing for more complex geometries to be represented which may be useful for placing devices inside waveguides or using the waveguide as a source for another device. Finally, the user experience of this implementation should be improved as it is remarkably easy to save in tens to hundreds of gigabytes of data inadvertently shifting the bottleneck away from memory to disk performance.
%\section{Appendix}
\label{sec:Appendix}

\subsection{Code Structure}
\label{subsec:code}
Code is broken up into logical modules, as is custom in Rust, which contain related aspects of the code. The file \verb|./src/main.rs| contains the `main' function that is built into a binary. The file \verb|./src/solver.rs| contains a high level interface for interacting with and bootstrapping the simulation. The file \verb|./src/geometry.rs| contains a structure that holds information relevant to the geometry of the simulation. Finally, the \verb|./src/engine.rs| contains all data and methods needed to evolve the simulation in time and contains much of the simulation code. All functions are commented using function comments in the source code which are automatically assembled into an interactive webpage containing all project documentation. Said documentation can be found under the \verb|./doc/| directory. As such project documentation can either be viewed by looking at the source code and/or viewing the interactive documentation pages by opening \verb|./doc/waveguide/index.html| with a web browser. The code can easily be compiled with cargo (the package manager that comes with Rust much like Pip for Python) using the command \verb|cargo build --release|. The compiled binary can then be executed by running \verb|./target/release/driver.exe|. This binary reads in data from \verb|config.toml| which contains all simulation parameters.


\bibliographystyle{IEEEtran}
\bibliography{cem_class_example_bib} %this should point to your bibliography file.

% that's all folks
\end{document}


