\section{Introduction}
\label{sec:intro}
Rectangular cavity resonators are used in a variety of applications ranging from filters to microwave energy storage devices \cite{pozar2011microwave}. Furthermore, rectangular cavity resonators are popular due to their simplistic design which can easily be constructed by placing shorting planes on the wave ports, thereby limiting energy loss to dielectric and wall surface conductivity\cite{pozar2011microwave}. This simplistic design allows for precise control of the unloaded quality factor (Q) and resonance frequency of these resonators by simply changing material properties and cavity length for example.

All wave phenomenon in waveguides and cavity resonators for a given frequency are derivable analytically from Maxwell's Equations. Maxwell's Equations, namely Faraday's and Amp\`{e}re's laws, can describe nearly all wave interactions in electromagnetics to a level of precision that few areas in physics can match \cite{rothlecnotes}. In 1966, Kane S. Yee proposed a temporally and spatially staggered grid which could be used to explicitly solve Maxwell's Equations using finite differences in the time domain \cite{yee}. The staggered Yee grid positions the electric and magnetic fields on the edges of spatially offset voxels at half integer time-steps\cite{yee}. This method resolved many of the erroneous solutions from previous finite-difference solutions as it constructs fields able to be integrated over a line\cite{rothlecnotes} much like the integral form of Maxwell's Equations.

Finite Difference Time Domain (FDTD) is well suited to model structures as $\mathrm{TE_{10}}$ waveguides and $\mathrm{TE_{101}}$ cavity resonators due to the similarity between device length and wavelength. This length symmetry allows for relatively few spatial and temporal `points' to be used to obtain a full-wave solution in these geometries. In addition to this, FDTD facilitates the modeling of wide-band pulses allowing for results from large patches of the frequency domain to be obtained from a single simulation which is ideal for studying device resonances.