\section{Introduction}
\label{sec:intro}

Waveguides are used in a plethora of applications ranging from transmitting microwave fields to acting as passive, low-pass filters \cite{pozar2011microwave}. While any cross section of a single conductor waveguide can support TE and TM modes, rectangular and circular cross sections are commonly chosen due to their ease of construction and analytic propagation characteristics. However, a limitation of rectangular waveguides is the limited bandwidth of their dominant mode which is less than an octave \cite{pozar2011microwave}. By adding a single or double ridge to the mouth of a waveguide, the cutoff frequency of the dominant mode can be reduced thus allowing for increased signal bandwidth \cite{pozar2011microwave}. This increased bandwidth comes at the cost of reduced power capacity due to the reduction in breakdown potential between the ridges \cite{pozar2011microwave} making them less ideal for High Power Microwave (HPM) devices.  

All wave phenomenon in an arbitrarily shaped, infinitely long waveguide at a given frequency are governed by the frequency domain Maxwell's Equations. Of these equations, Faraday's and Amp\`{e}re's laws, can be manipulated to create Helmholtz wave equations which capture nearly all electromagnetic wave phenomenon to a high degree of accuracy \cite{rothlecnotes}. The Finite Element Method (FEM) converges on the analytic solution of the wave equation by approximating a weak form of the Helmholtz equations over a finite set of elements within the simulation domain using weighted residuals. To convert the full wave equation to its weak form, the Galerkin method is employed for which the weighting functions are identical to continuous basis functions as is common in Computational Electromagnetics (CEM) Finite Element Analyis (FEA) \cite{rothlecnotes}. In the case of an arbitrarily shaped, infinitely long waveguide, the full-field, frequency-domain solutions can be obtained by solving for the fields in a cross sectional slice of the waveguide. FEM operates on non-uniform, conformal meshes which allows for arbitrary waveguide cross sections to be modeled without stairstepping error unlike that of the structured meshes of Finite-Difference Time-Domain (FDTD). In addition to this, FEM allows for full three dimensional solutions of such a waveguide to be obtained by only solving for a representative two dimensional slice making FEM an ideal choice for analyzing homogenous waveguides.

The development and results of this work are laid out as follows. Section \ref{sec:mathmod} contains derivations of the Galerkin weak forms of the Helmholtz wave equations for both the TE and TM modes as well as the formation of the FEM matrices via the assembly process. Section \ref{sec:numres} contains a verification of the model with analytic data for a square waveguide, an analysis and discussion of propagation in circular waveguides, as well as a comparison of rectangular waveguides to their ridged counterparts. Finally, Section \ref{sec:conclusion} contains closing remarks regarding the analysis and potential future work.