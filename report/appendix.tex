\section{Appendix}
\label{sec:Appendix}

\subsection{Code Structure}
\label{subsec:code}
Code is broken up into logical modules, as is custom in Rust, which contain related aspects of the code. The file \verb|./src/main.rs| contains the `main' function that is built into a binary. The file \verb|./src/solver.rs| contains a high level interface for interacting with and bootstrapping the simulation. The file \verb|./src/geometry.rs| contains a structure that holds information relevant to the geometry of the simulation. Finally, the \verb|./src/engine.rs| contains all data and methods needed to evolve the simulation in time and contains much of the simulation code. All functions are commented using function comments in the source code which are automatically assembled into an interactive webpage containing all project documentation. Said documentation can be found under the \verb|./doc/| directory. As such project documentation can either be viewed by looking at the source code and/or viewing the interactive documentation pages by opening \verb|./doc/waveguide/index.html| with a web browser. The code can easily be compiled with cargo (the package manager that comes with Rust much like Pip for Python) using the command \verb|cargo build --release|. The compiled binary can then be executed by running \verb|./target/release/driver.exe|. This binary reads in data from \verb|config.toml| which contains all simulation parameters.